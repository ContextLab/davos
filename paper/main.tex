%% 
%% Copyright 2007, 2008, 2009 Elsevier Ltd
%% 
%% This file is part of the 'Elsarticle Bundle'.
%% ---------------------------------------------
%% 
%% It may be distributed under the conditions of the LaTeX Project Public
%% License, either version 1.2 of this license or (at your option) any
%% later version.  The latest version of this license is in
%%    http://www.latex-project.org/lppl.txt
%% and version 1.2 or later is part of all distributions of LaTeX
%% version 1999/12/01 or later.
%% 
%% The list of all files belonging to the 'Elsarticle Bundle' is
%% given in the file `manifest.txt'.
%% 

%% Template article for Elsevier's document class `elsarticle'
%% with numbered style bibliographic references
%% SP 2008/03/01

\documentclass[preprint,12pt, a4paper]{elsarticle}

%% Use the option review to obtain double line spacing
%% \documentclass[authoryear,preprint,review,12pt]{elsarticle}

%% For including figures, graphicx.sty has been loaded in
%% elsarticle.cls. If you prefer to use the old commands
%% please give \usepackage{epsfig}

%% The amssymb package provides various useful mathematical symbols
\usepackage{amssymb}
%% The amsthm package provides extended theorem environments
%% \usepackage{amsthm}

%% The lineno packages adds line numbers. Start line numbering with
%% \begin{linenumbers}, end it with \end{linenumbers}. Or switch it on
%% for the whole article with \linenumbers.
\usepackage{lineno}

\usepackage{float}
\restylefloat{table}

\usepackage[hidelinks]{hyperref}
\usepackage{breakurl}

\journal{SoftwareX}

\begin{document}
\begin{frontmatter}

\title{\texttt{davos}: The Python package smuggler}
\author{Paxton C. Fitzpatrick}
\author{Jeremy R. Manning\corref{cor}}
\ead{Jeremy.R.Manning@Dartmouth.edu}
\cortext[cor]{Corresponding author}
\address{Department of Psychological and Brain Sciences\\Dartmouth College, Hanover, NH 03755}

% ------------------------------------------------------ ABSTRACT ------------------------------------------------------

\begin{abstract}
% Text of abstract 
% Ca. 100 words

\end{abstract}


\begin{keyword}
Python \sep Jupyter Notebook \sep Google Colaboratory \sep Reproducibility \sep Package management \sep Pip install
\end{keyword}

\end{frontmatter}


% ------------------------------------------------------ METADATA ------------------------------------------------------
\section*{Required Metadata}

\section*{Current code version}


\begin{table}[H]
\begin{tabular}{|l|p{6.5cm}|p{6.5cm}|}
\hline
\textbf{Nr.} & \textbf{Code metadata description} & \textbf{Please fill in this column} \\
\hline
C1 & Current code version &  v0.1.0 \\
\hline
C2 & Permanent link to code/repository used for this code version & \url{https://github.com/ContextLab/davos} \\
\hline
C3 & Code Ocean compute capsule & \\
\hline
C4 & Legal Code License & MIT \\
\hline
C5 & Code versioning system used & git \\
\hline
C6 & Software code languages, tools, and services used & Python, JavaScript, pip, IPython, Jupyter, Ipykernel, PyZMQ. Additional tools used for tests: pytest, Selenium, Requests, mypy, GitHub Actions \\
\hline
C7 & Compilation requirements, operating environments \& dependencies & Dependencies:~Python$>$=3.6, packaging, setuptools.~Supported OSes: MacOS, Linux, Unix-like.~Supported IPython environments: Jupyter notebooks, JupyterLab, Google Colaboratory, Binder, IDE-based notebook editors. \\
\hline
C8 & If available Link to developer documentation/manual & \url{https://github.com/ContextLab/davos\#readme} \\
\hline
C9 & Support email for questions & contextualdynamics@gmail.com \\
\hline
\end{tabular}
\caption{Code metadata}
\label{}
\end{table}

\linenumbers
%The permanent link to code/repository or the zip archive should include the following requirements: 

%README.txt and LICENSE.txt.

%Source code in a src/ directory, not the root of the repository.

%Tag corresponding with the version of the software that is reviewed.

%Documentation in the repository in a docs/ directory, and/or READMEs, as appropriate.


% --------------------------------------------- MOTIVATION & SIGNIFICANCE ----------------------------------------------
\section{Motivation and significance}

%Introduce the scientific background and the motivation for developing the software.

%Explain why the software is important, and describe the exact (scientific) problem(s) it solves.

%Indicate in what way the software has contributed (or how it will contribute in the future) to the process of scientific discovery; if available, this is to be supported by citing a research paper using the software.

%Provide a description of the experimental setting (how does the user use the software?).

%Introduce related work in literature (cite or list algorithms used, other software etc.).


% ------------------------------------------------ SOFTWARE DESCRIPTION ------------------------------------------------
\section{Software description}
% Describe the software in as much as is necessary to establish a vocabulary needed to explain its impact. 


\subsection{Software Architecture}
% Give a short overview of the overall software architecture; provide a pictorial component overview or similar (if possible). If necessary provide implementation details.


\subsection{Software Functionalities}
% Present the major functionalities of the software.


\subsection{Sample code snippets analysis (optional)}


% ----------------------------------------------- ILLUSTRATIVE EXAMPLES ------------------------------------------------
\section{Illustrative Examples}
% Provide at least one illustrative example to demonstrate the major functions.

% Optional: you may include one explanatory video that will appear next to your article, in the right hand side panel. (Please upload any video as a single supplementary file with your article. Only one MP4 formatted, with 50MB maximum size, video is possible per article. Recommended video dimensions are 640 x 480 at a maximum of 30 frames/second. Prior to submission please test and validate your .mp4 file at $ http://elsevier-apps.sciverse.com/GadgetVideoPodcastPlayerWeb/verification$. This tool will display your video exactly in the same way as it will appear on ScienceDirect.).


% ------------------------------------------------------- IMPACT -------------------------------------------------------
\section{Impact}

%\textbf{This is the main section of the article and the reviewers weight the description here appropriately}

%Indicate in what way new research questions can be pursued as a result of the software (if any).

%Indicate in what way, and to what extent, the pursuit of existing research questions is improved (if so).

%Indicate in what way the software has changed the daily practice of its users (if so).

%Indicate how widespread the use of the software is within and outside the intended user group.

%Indicate in what way the software is used in commercial settings and/or how it led to the creation of spin-off companies (if so).


% ---------------------------------------------------- CONCLUSIONS -----------------------------------------------------
\section{Conclusions}
% Set out the conclusion of this original software publication.


\section*{Funding}
Our work was supported in part by NSF grant number 2145172 to J.R.M. The content is solely the responsibility of the authors and does not necessarily represent the official views of our supporting organizations.


\section*{Declaration of Competing Interest}
We wish to confirm that there are no known conflicts of interest associated with this publication and there has been no significant financial support for this work that could have influenced its outcome.


\section*{Acknowledgements}


%---------------------------------------------------- BIBLIOGRAPHY -----------------------------------------------------
\bibliographystyle{elsarticle-num} 
\bibliography{main.bib}

\end{document}

