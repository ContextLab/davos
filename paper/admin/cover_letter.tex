\title{cover letter}
%
% See http://texblog.org/2013/11/11/latexs-alternative-letter-class-newlfm/
% and http://www.ctan.org/tex-archive/macros/latex/contrib/newlfm
% for more information.
%
\documentclass[11pt,stdletter,orderfromtodate,sigleft]{newlfm}
\usepackage{hyperref, pxfonts, geometry}

  \setlength{\voffset}{0in}

\newlfmP{dateskipbefore=0pt}
\newlfmP{sigsize=20pt}
\newlfmP{sigskipbefore=10pt}
 
\newlfmP{Headlinewd=0pt,Footlinewd=0pt}
 
\namefrom{\vspace{-0.3in}Jeremy R. Manning}
\addrfrom{
	Dartmouth College\\
     Department of Psychological \& Brain Sciences\\
    HB 6207 Moore Hall\\
	Hanover, NH  03755}
 
\addrto{}
\dateset{\today}
 
\greetto{To the editors of \textit{Software X}:}


 
\closeline{Sincerely,}
%\usepackage{setspace}
%\linespread{0.85}
% The cover letter should explain the importance of the work, and why you consider it appropriate for the diverse readership of . 
% OPTIONAL. Provide the name and institution of reviewers you would like to recommend and/or people you would like to be excluded from peer review (explaining why).

\begin{document}
\begin{newlfm}

  We have enclosed our manuscript entitled \textit{\texttt{davos}: a
    Python package ``smuggler'' for constructing lightweight
    reproducible notebooks} to be considered for publication as
  an \textit{Article}.

  Our manuscript describes a new Python package, \texttt{davos}.  When
  used in combination with a notebook-based Python project, the
  \texttt{davos} library provides tools for specifying and
  automatically installing the correct versions of the project's
  dependencies.  Our library also ensures that the correct versions of
  those dependencies are in use any time the notebook's code is
  executed.  This enables researchers to share a complete reproducible
  copy of their code within a single Jupyter notebook (.ipynb) file.

  Broadly, we designed the \texttt{davos} library to target a ``sweet
  spot'' along a continuum of existing approaches to facilitating reproducible
  code-based research.  At one end of this continuum, ``lightweight''
  approaches entail simply sharing raw code (e.g., plain-text python
  scripts) or Jupyter notebooks (e.g., JSON files that comprise a mix
  of text, code, and embedded media).  These lightweight solutions
  benefit from very low setup costs (which increase accessibility),
  but they typically do not make any attempt to manage or constrain
  the computing environment in which the shared code is executed.
  At best, when dependencies are missing on the end user's system, shared code
 may fail to run entirely.  And when the \textit{versions} of a
 project's dependencies differ across the original author's system and
 the end user's system, shared code may (at worst) behave in
 unexpected ways or even cause damage.

 At the other end of this continuum, ``heavyweight'' approaches entail
 simulating or replicating, to varying depths, the original computing
 environment that the shared code was developed for.  For example,
 virtual environments, containerized systems, and virtual machines
 reproduce (respectively) a complete Python environment, operating system, and/or
 full hardware simulation of the original environment.  Each of these
 systems guarantees, to varying degrees, that shared code will behave
 as expected for the end user.  A downside to these approaches is that
 they are often effort- and/or resource-intensive, since they require
 installing additional tools (e.g., Anaconda, Docker, machine
 emulators, etc.) and
 configuration files (e.g., environment configuration files,
 Dockerfiles, system images, etc.), alongside the to-be-shared code
 itself.

 The \texttt{davos} library is lightweight in the sense that it does
 not require any setup beyond that required to run standard Jupyter
 notebooks.  But \texttt{davos} also provides infrastructure for
 precisely controlling project dependencies in a way that can easily
 be embedded into standard notebooks.  This provides a complete system
 for sharing reproducible code inside of a standard notebook file.
  
 Beyond its intended primary role in facilitating reproducible
 research, \texttt{davos} is also useful in pedagogical settings
 (e.g., courses that involve programming in notebook-based
 environments), or when putting together lightweight notebook-based
 demonstrations.

 Thank you for considering our manuscript, and we
 hope you will find it suitable for publication in \textit{Software
   X}.


\end{newlfm}
\end{document}
